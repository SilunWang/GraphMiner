% You should title the file with a .tex extension (hw1.tex, for example)
\documentclass[11pt]{article}

\usepackage{amsmath}
\usepackage{amssymb}
\usepackage{fancyhdr}
\usepackage{enumerate}
\usepackage{listings}
\usepackage{color}
\usepackage{titlesec}

\titleformat*{\section}{\large\bfseries}

\definecolor{dkgreen}{rgb}{0,0.6,0}
\definecolor{gray}{rgb}{0.5,0.5,0.5}
\definecolor{mauve}{rgb}{0.58,0,0.82}

\lstset{frame=tb,
  language=Python,
  aboveskip=3mm,
  belowskip=3mm,
  showstringspaces=false,
  columns=flexible,
  basicstyle={\small\ttfamily},
  numbers=none,
  numberstyle=\tiny\color{gray},
  keywordstyle=\color{blue},
  commentstyle=\color{dkgreen},
  stringstyle=\color{mauve},
  breaklines=true,
  breakatwhitespace=true,
  tabsize=3
}

\oddsidemargin0cm
\topmargin-2cm     %I recommend adding these three lines to increase the 
\textwidth16.5cm   %amount of usable space on the page (and save trees)
\textheight23.5cm  

\newcommand{\question}[2] {\vspace{.25in} \hrule\vspace{0.5em}
\noindent{\bf #1: #2} \vspace{0.5em}
\hrule \vspace{.10in}}
\renewcommand{\part}[1] {\vspace{.10in} {\bf (#1)}}

\newcommand{\name}{Silun Wang}
\newcommand{\andrew}{silunw@andrew.cmu.edu}
\newcommand{\myname}{Yuwei Zhang}
\newcommand{\myandrew}{yuweiz1@andrew.cmu.edu}
\newcommand{\myhwnum}{1}

\setlength{\parindent}{0pt}
\setlength{\parskip}{5pt plus 1pt}
 
\pagestyle{fancyplain}
\lhead{\fancyplain{}{\textbf{Project Phase\myhwnum}}}      % Note the different brackets!
\rhead{\fancyplain{}{\andrew\\ \myandrew}}
\chead{\fancyplain{}{15-826 }}

\begin{document}

\medskip                        % Skip a "medium" amount of space
                                % (latex determines what medium is)
                                % Also try: \bigskip, \littleskip

\thispagestyle{plain}
\begin{center}                  % Center the following lines
{\Large 15-826 Project Phase\myhwnum} \\[10pt]
\name \\
\andrew \\
\myname \\
\myandrew \\

\end{center}
\begin{lstlisting}
#Task 0: kcore
def gm_kcore ():
    # compute coreness of each node
    print "Computing kcore..."
    
    cur = db_conn.cursor()
    GM_TABLE_DUP = "GM_TABLE_DUP"
    GM_KCORE_TMP = "GM_KCORE_TMP"
    gm_sql_table_drop_create(db_conn, GM_KCORE, "node_id integer, \
                                 coreness integer")
    gm_sql_table_drop_create(db_conn, GM_TABLE_DUP, "src_id integer, dst_id integer")

    cur.execute("insert into %s" % GM_TABLE_DUP + " select src_id, dst_id from %s" %GM_TABLE)
    db_conn.commit()

    cur.execute("create index on %s (src_id, dst_id)" % GM_TABLE_DUP)
    db_conn.commit()

    k = 1
    while True:
        # each time we pick out elements with less than k neighbors and output them with coreness k
        # delete from the original table whose neighbor are those elements
        # if we get no such elements, we increase k
        # until we get no elements left in the orginal table

        gm_sql_table_drop_create(db_conn, GM_KCORE_TMP, "src_id integer, \
                                 neighbor integer")
        cur.execute ("insert into %s" % GM_KCORE_TMP + 
                     " select src_id, count(*) as neighbor from %s" % GM_TABLE_DUP +
                     " group by src_id having count(*) <= %d" % k)
        cur.execute("create index on %s (src_id)" % GM_KCORE_TMP)
        db_conn.commit()
        cur.execute("select count(*) from %s" % GM_KCORE_TMP)
        val = cur.fetchone()[0]
        if val == 0:
            k += 1
            continue

        cur.execute ("INSERT INTO %s" % GM_KCORE +
                                 " SELECT src_id , %d" % k + " as coreness from %s" %GM_KCORE_TMP)
        db_conn.commit()

        cur.execute("delete from %s"%GM_TABLE_DUP + " where src_id in (select src_id from %s)"%GM_KCORE_TMP)
        cur.execute("delete from %s"%GM_TABLE_DUP + " where dst_id in (select src_id from %s)"%GM_KCORE_TMP)
        db_conn.commit()  

        cur.execute("select count(*) from %s"%GM_TABLE_DUP)
        val = cur.fetchone()[0]
        if val == 0:
            break

    gm_sql_table_drop(db_conn, GM_TABLE_DUP)
    gm_sql_table_drop(db_conn, GM_KCORE_TMP)

    cur.close()
\end{lstlisting}
\end{document}

